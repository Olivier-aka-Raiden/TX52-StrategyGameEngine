\documentclass[a4paper,10pt]{book}

\usepackage[utf8]{inputenc}
\usepackage[english]{babel}
\usepackage{graphicx}
\usepackage{float}
\usepackage{geometry}
\geometry{margin=3cm}

\newcommand{\quotes}[1]{``#1''}
\newcommand\tab[1][1cm]{\hspace*{#1}}

%opening
\title{\textbf{TX52 - scientific project report }\\
  Create a Real-Time Strategy Game Engine with SARL, an agent oriented language}
\author{~\\
  \textbf{Olivier Villequey, Romain Dulieu} }

\begin{document}

\maketitle

\tableofcontents

\chapter{Project Definition}

\section {Introduction}

~

The subject of our project is to create a Real-Time Strategy (RTS) game engine using SARL language, an agent oriented language based on java.
Firstly, we have to define what is a RTS game and what are the important specificities or components of this kind of games :
\begin{itemize}
 \item Real-time strategy : include an enormous number of possible game actions that can be executed at any given time. The result or effect of any action is unknown
  and changes as the time goes by. Furthurmore, RTS games usually manipulate a large number of units, controlled by an AI, which is an important aspect of our project. 
 \item We consider that RTS games include only partially obervable environments. It means that one aspect of the strategy is to collect informations by exploring the environment.
 \item A map : an environment composed of different objects static (walls, trees, cliffs, etc.) or dynamic (units) with specific 2D/3D environment features such as the topography etc.
 \item Different types of AIs : strategic level or global AI that takes abstract decision making to achieve the main goal : deciding the size of the force necessary to move/attack a position, the kind of forces needed etc. 
  On the other hand, the tactical level or local AI concerns more concrete decisions (handle the actions : moving,attacking,producing etc.) and is responsible for achieving the objectives defined at the strategic level.
\end{itemize}

~

\section {State of the art}

~

First of all, as the gaming tools and games are developed in private companies it is difficult to get recent public papers about the actual technologies used in RTS games. 
That's why our sources may be a little bit old but it is easier to find public sources from a couple of years ago.

~

\subsection{RTS history}

~

Real-time strategy (RTS) games are known to be one of the
most  complex  game  genres  for  humans  to  play,  as  well  as
one  of  the  most  difficult  games  for  computer  AI  agents  to
play well.  To tackle the task of applying AI to RTS games,
recent techniques have focused on a divide-and-conquer approach, 
splitting the game into strategic components, and developing 
separate systems to solve each. This trend gives rise
to  a  new  problem:   how  to  tie  these  systems  together  into
a functional real-time strategy game playing agent.

~

Traditional games such as Chess and Go have for centuries
been  regarded  as  the  most  strategically  difficult  games  to
play at a top level. High-level play involves complex strategic
decisions  based  on  knowledge  obtained  through  study
and  training,  combined  with  online  analysis  of  the  pieces
on the current board.  Top players are able to “look ahead”
a dozen or more moves into the future to decide on an
action, often under strict time constraints, with clocks for each
player ticking away as they think make their decision.  Let
us now imagine a genre of game in which the playing field
is 256 times as large, contains up to several hundred pieces
per player, with pieces able to be created or destroyed at any
moment.  On top of this, players may move any number of
pieces simultaneously in real-time, with the only limit being
their own dexterity.   What we have just described is a
real-time strategy (RTS) game, which combines the complex
strategic elements of traditional games with the real-time 
actions of a modern video game.

~

A relatively new genre, the first RTS games started to appear
in  the  early  1990s  with  titles  such  as  Dune  II,  
WarCraft,  and Command and Conquer.   Originally introduced
as a single-player war simulation, their popularity exploded
as the internet allowed for players to compete against each
other in multiplayer scenarios. With the creation of StarCraft
in 1998, RTS games had reached a level of strategy unseen
in other video game genres.

~

\subsection{Environment}

~

RTS games take place on a map, composed of a finite or infinite number of cells with a position organized in a grid. On this basis,
you can choose to create a continue or discontinue environment space. Most of the RTS are still based on a two-dimensional map even 
in 3D engines. In fact, newer games didn't innovate much on the initial concept but tend to emphasize more on the basic RTS elements
such as higher unit cap, more unit types, larger maps, etc.
Environments can implement climate changes, different types of terrain that impact certain types of movement etc.

~


\subsection{Path Finding}

~

Path finding is the ability for the agent to find his way from his position to a destination by taking the shortest path and of course avoiding the obstacles between him and the destination position.
In most commercial RTSs, the solution for the path finding is using A* over a navigation mesh (commonly referred to as a \quotes{navmesh}). Navmesh is used to create the nodes of our graph by creating polygons	on the surface area where the agents can move. This can be hard coded and so labor intensive or it is possible to create an algorithm that generate the navmesh from a given map.
When the navmesh is created, you can apply A* algorithm on the graph to determine the shortest path.
To do so, here is a short explanation of how A* algorithm works:
\begin{itemize}
 \item Create two lists : CLOSED for the nodes already evaluated and OPEN for the ones to be evaluated. Add the starting node to the OPEN list.
 \item A loop : select the lowest cost node in open and move it to the CLOSED list
 \item \tab if the node is the target then it's over
 \item \tab for each neighbour of the selected node, if it is in CLOSED skip to the next neighbour
 \item \tab \tab if the new path to neighbour is shorter or neighbour is not in OPEN then set his new path cost and set the current node as its parent
 \item \tab \tab if neighbour is not in OPEN add neighbour to OPEN.
 \item	\tab end of for
 \item end of loop
\end{itemize}
             
~

\subsection{AIs in RTS}

~

Inspired by  military command structures,  tasks are partitioned among modules
by their intuitive strategic meaning (combat, economy, etc.),
with vertical communication being performed on a “need to
know” basis.  High level strategy decisions are made by the
global AI by compiling all known information about
the current game state.  Commands are then given to local AI
which are directly in charge of completing the low-level task.

~

\textbf{References} 
\textit{\\Incorporating Search Algorithms into RTS Game Agents},
David Churchill and Michael Buro

\section {Models}

~

\subsection{Architecture}

~

Our program is divided in three major parts :
\begin{itemize}
 \item The environment, written in Java, that contains the world we created as well as the Jbox2D world that solve the physics
 problems. Both worlds will coexist in the main class Environment.
 \item The agent part, written in SARL, with all agents : the Environment Agent, the main controller of the game engine. It
 communicates with the Environment, all the units and the GUI. It coordinates their actions and leads the main cycle of the game.
 Then, there is the Units Agents with their bodies in the environment and the possibility to act with some behaviours corresponding to
 their perceptions.
 \item The Graphical User Interface, written in Java using Swing libraries. We didn't choose the best solution to do the GUI as
 we don't have much experience with JavaFX and so we chose to use Swing because it was easier for us and finally the GUI wasn't the most
 important objective of our project.
\end{itemize}

~

\subsection{UML}

~

Firstly, the diagram below shows the class diagram of the environment :

~

\begin{figure}[!ht]
 \parshape1 -2.5cm 21cm
 \centering
 \includegraphics[scale=0.5]{TX52UML}
 \caption{Environment UML}
\end{figure}
\parshape0

~

The center class is of course Environment with both Jbox2D world and the world we created. it also contain the list of events
from the listeners of the environment, more precisely the GUI in our project. Jbox2D contains the objects and is used to handle
the physics whereas EnvMap is the world we made with a tree algorithm that contains all the objects, methods to add or remove objects
and the list of bodies of all the units.

~

All the objects are Environment Objects, they can be Static (walls, obstacles etc) or Dynamic (units). AgentBody is an example of a unit
that we added with a specific perceptionDistance, attack speed, life and also its team.

~

Finally, within his perception range, agents have a list of Perceivable which contains the ID of the object and several information that
will be used to decide its following influence. Finally, EnvironmentChangeQuery is used to communicate with the environment agent and transmit
the next influence.

~

\newpage

The following UML shows the agent part mostly written in SARL :

~

\begin{figure}[!ht]
 \parshape1 -2.5cm 21cm
 \centering
 \includegraphics[scale=0.6]{TX52AgentUML}
 \caption{Agent UML}
\end{figure}
\parshape0

~

\section {Application}

~

\section {Performance}


\end{document}
